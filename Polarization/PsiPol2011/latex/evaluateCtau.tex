\documentclass[11pt,slidescentered,red,compress,handout,hyperref={bookmarks=true},mathseriftable]{beamer}
\mode<presentation>{
	\usetheme{Madrid}  % Madrid, Antibes, Frankfurt, Montpellier
	\useinnertheme{circles}
	%\usecolortheme{beaver}
	\usefonttheme[onlymath]{serif}
	\setbeamercovered{transparent}
	%\setbeamertemplate{footline}[frame number]{}
}

\usepackage{tikz}
%\usepackage{graphicx}
\usepackage{amssymb}
\usepackage{subfigure}
\usepackage{psfrag}
\usepackage{multimedia}
\usepackage{pgfpages}
%\usepackage{hepnicenames}
\usepackage[overlay,absolute]{textpos}
%\usepackage[overlay,showboxes,absolute]{textpos}
\usepackage{color}
\usepackage{hyperref}
\usepackage{multirow}
\usepackage{graphics}
%\TPGrid{100}{100}
\setlength{\TPHorizModule}{0.1in}
\setlength{\TPVertModule}{0.1in}
\setbeamercolor{postit}{fg=black,bg=yellow}
\beamertemplatenavigationsymbolsempty
%Purdue Yellow
%albatross | beaver | beetle |
%	crane | default | dolphin |
%	dove | fly | lily | orchid |
%	rose |seagull | seahorse |
%	sidebartab | structure |
%	whale | wolverine
%\usecolortheme[RGB={255,203,100}]{crane}
%\usetheme{AnnArbor}
\usecolortheme{seahorse}
\useinnertheme{circles}
%\useoutertheme{tree}
%CMS Blue
%\usecolortheme[RGB={141,192,255}]{crane}
%\usecolortheme{crane}
\usefonttheme{professionalfonts}
%\useinnertheme{rounded}
\setbeamertemplate{frametitle}[default][center]
\addtobeamertemplate{frametitle}{}{
%  \begin{textblock}{5}(0,0)
 %   \includegraphics[scale=0.068]{CMSlogo}
 % \end{textblock}
%  \begin{textblock}{5}(41.2,0.6)
%    \includegraphics[scale=0.05]{HEPHY_Logo}
%  \end{textblock}
}
\setbeamertemplate{sidebar right}{}
\setbeamercolor*{author in head/foot}{parent=palette quaternary}
\setbeamercolor*{title in head/foot}{parent=palette secondary}
\setbeamercolor*{date in head/foot}{parent=palette primary}
\setbeamertemplate{footline}
{
	\leavevmode%
		\hbox{%
			\begin{beamercolorbox}[wd=.55\paperwidth,ht=2.25ex,dp=1ex,left]{author in head/foot}%
				\usebeamerfont{author in head/foot}~\insertsection
			\end{beamercolorbox}%
			\begin{beamercolorbox}[wd=.30\paperwidth,ht=2.25ex,dp=1ex,left]{title in head/foot}%
				\usebeamerfont{title in head/foot}~\hfill polarization \hfill
				%\usebeamerfont{title in head/foot}~\hfill Linlin Zhang\hfill
			\end{beamercolorbox}%
			\begin{beamercolorbox}[wd=.15\paperwidth,ht=2.25ex,dp=1ex,right]{date in head/foot}%
				\usebeamerfont{date in head/foot}~
		 	   \insertframenumber{} / \inserttotalframenumber{}\hspace*{2ex}
			\end{beamercolorbox} }%
	\vskip0pt%
}

%\institute{\includegraphics[scale=0.09]{Figures/CMSlogo}
	\institute{\includegraphics[scale=0.09]{Figures/pkulogo}
		%\includegraphics[scale=0.065]{Figures/HEPHY_Logo}
	}


\newcommand{\pT}{p_\mathrm{T}}
\newcommand{\Jpsi}{\mathrm{J}/\psi}
\newcommand{\UpsOne}{\Upsilon(1S)}
\newcommand{\UpsTwo}{\Upsilon(2S)}
\newcommand{\UpsThree}{\Upsilon(3S)}
\newcommand{\UpsN}{\Upsilon(nS)}
\newcommand{\abseta}{\left |  \eta \right |}
\newcommand{\absy}{\left |  y \right |}
\newcommand{\costh}{\cos\vartheta}
\newcommand{\inpb}{\text{pb}^{-1}}
\newcommand{\infb}{\text{fb}^{-1}}
\newcommand{\TM}{\mu^{{\scriptscriptstyle TM}}}
\newcommand{\HLTM}{\mu^{{\scriptscriptstyle HLT}}}
\newcommand{\pxp}{ p_x^{\mu^{+} } }
\newcommand{\pyp}{ p_y^{\mu^{+} } }
\newcommand{\pzp}{ p_z^{\mu^{+} } }
\newcommand{\pxm}{ p_x^{\mu^{-} } }
\newcommand{\pym}{ p_y^{\mu^{-} } }
\newcommand{\pzm}{ p_z^{\mu^{-} } }
\newcommand{\ppvec}{\vec{p}^{\;\mu^{+} } }
\newcommand{\pmvec}{\vec{p}^{\;\mu^{-} } }
\newcommand{\mmumu}{m_{\mu\mu}}
\newcommand{\ljpsi}{l_{\Jpsi}}
\newcommand{\lpsi}{\ell_{\psi}}
\newcommand{\mpsi}{M_{\psi}}
%lambdas:
\newcommand{\lamth}{\lambda_\vartheta}
\newcommand{\lamph}{\lambda_\varphi}
\newcommand{\lamthph}{\lambda_{\vartheta\varphi}}
\newcommand{\lamtilde}{\tilde{\lambda}}
\newcommand{\lam}{\mbox{\boldmath$\lambda$}}
\newcommand{\lamthCS}{\lambda^{\scriptscriptstyle CS}_\vartheta}
\newcommand{\lamphCS}{\lambda^{\scriptscriptstyle CS}_\varphi}
\newcommand{\lamthphCS}{\lambda^{\scriptscriptstyle CS}_{\vartheta \varphi}}
\newcommand{\lamtildeCS}{\tilde{\lambda}^{\scriptscriptstyle CS}}
\newcommand{\lamthHX}{\lambda^{\scriptscriptstyle HX}_\vartheta}
\newcommand{\lamphHX}{\lambda^{\scriptscriptstyle HX}_\varphi}
\newcommand{\lamthphHX}{\lambda^{\scriptscriptstyle HX}_{\vartheta \varphi}}
\newcommand{\lamtildeHX}{\tilde{\lambda}^{\scriptscriptstyle HX}}
\newcommand{\lamthPX}{\lambda^{\scriptscriptstyle PX}_\vartheta}
\newcommand{\lamphPX}{\lambda^{\scriptscriptstyle PX}_\varphi}
\newcommand{\lamthphPX}{\lambda^{\scriptscriptstyle PX}_{\vartheta \varphi}}
\newcommand{\lamtildePX}{\tilde{\lambda}^{\scriptscriptstyle PX}}

\newcommand{\sigsyst}{\sigma_{syst}}
\newcommand{\sigstat}{\sigma_{stat}}
\newcommand{\tnp}{T\&P}

% 'SIX' lambda plot positions
\newcommand{\SIXplotOneY}{12}
\newcommand{\SIXplotTwoY}{24}
\newcommand{\SIXplotOneX}{1}
\newcommand{\SIXplotTwoX}{17}
\newcommand{\SIXplotThreeX}{33}
\newcommand{\SIXplotSize}{16}

% Lambda label positions
\newcommand{\LabelSIXplotOneY}{14}
\newcommand{\LabelSIXplotTwoY}{26}
\newcommand{\LabelSIXplotOneX}{6}
\newcommand{\LabelSIXplotTwoX}{22}
\newcommand{\LabelSIXplotThreeX}{38}
\newcommand{\LabelSIXplotSize}{5}
\newcommand{\LabelSIXplotScriptSize}{\tiny}

% Lambda label positions in Results plots
\newcommand{\ResultLabelSIXplotOneY}{13.5}
\newcommand{\ResultLabelSIXplotTwoY}{25.5}
\newcommand{\ResultLabelSIXplotOneX}{13.5}
\newcommand{\ResultLabelSIXplotTwoX}{29.5}
\newcommand{\ResultLabelSIXplotThreeX}{45.5}
\newcommand{\ResultLabelSIXplotSize}{5}
\newcommand{\ResultLabelSIXplotScriptSize}{\tiny}

% Figure directories
\newcommand{\FittingRltsOneS}{../Psi1S/Fit/parameter/evaluateCtau/}
\newcommand{\FittingRltsTwoS}{../Psi2S/Fit/parameter/evaluateCtau/}

%%%%% Include this definition (Joao) %%%%%%%%%%%%
\newlength{\eqboxstorage}
\newcommand{\eqbox}[1]{
\setlength{\eqboxstorage}{\fboxsep}
\setlength{\fboxsep}{6pt}
\boxed{#1}
\setlength{\fboxsep}{\eqboxstorage}
}

%%%%%%%%%%%%% 
%%%Colors%%%%
%%%%%%%%%%%%%
%black= body 
%blue=structure 
%red=alert 
%green=example

%%%%%%%%%%%%%%% 
%%%FontSize%%%%
%%%%%%%%%%%%%%%
%\tiny  5pt 
%\scriptsize 7pt 
%\footnotesize 8pt 
%\small  9pt 
%\normalsize 10pt 
%\large  12pt 
%\Large  14.4pt 
%\LARGE  17.28pt 
%\huge  20.74pt 
%\Huge  24.88pt 

\begin{document}

%\begin{frame}[plain]
%  \titlepage
%\end{frame}
%
%\begin{frame}[t]{\small \bf Outline}{}
%\tableofcontents
%\end{frame}

\section{Evaluate $\lpsi$ cut (P region)} 
\begin{frame}[t]{\small \bf Evaluate $\lpsi$ cut (P region)}{}
\begin{textblock}{48}(1,5.25)
\begin{itemize}
\scriptsize \item I) get $\sigma$ of prompt p.d.f, by fitting the trend changes as $p_{T}$
\scriptsize \item II) define P region as a symmetric range around zero with nSigma
\scriptsize \item III) Evaluate Prompt, Non-prompt, and Background fractions in $\lpsi$ window [-nSigma*$\sigma$, nSigma*$\sigma$] \\
\end{itemize}
\end{textblock}
\begin{textblock}{48}(1,13)
\scriptsize $\sigma$ changes vs. $p_{T}$ for all rapidity bins: \structure{1S (left), 2S (right)}
\hspace*{10pt} \includegraphics[width=0.45\textwidth]{\FittingRltsOneS PR_2.5sigma/rms.pdf}
\hspace*{10pt} \includegraphics[width=0.45\textwidth]{\FittingRltsTwoS PR_2sigma/rms.pdf}
\end{textblock}
\end{frame}

%%%%%%%%%%%%%%% 1S
\begin{frame}[t]{\small \bf fractions in P region, \structure{$\psi$(1S), nSigma = 2.5} }{}
\begin{textblock}{48}(1,5.25)
\begin{itemize}
\scriptsize \item  $|$y$|$ $<$ 0.6(left) and 0.6 $<$ $|$y$|$ $<$ 1.2(right)
\end{itemize}
\end{textblock}
\begin{textblock}{48}(1,6)
\hspace*{10pt} \includegraphics[width=0.45\textwidth]{\FittingRltsOneS PR_2.5sigma/fraction_rap1.pdf}
\hspace*{10pt} \includegraphics[width=0.45\textwidth]{\FittingRltsOneS PR_2.5sigma/fraction_rap2.pdf}
\end{textblock}
\end{frame}

%%%%%%%%%%% errors on fraction
\begin{frame}[t]{\small \bf fractions in P region, \structure{Errors $\psi$(1S), nSigma = 2.5} }{}
\begin{textblock}{48}(1,5.25)
\begin{itemize}
\scriptsize \item  $|$y$|$ $<$ 0.6(left) and 0.6 $<$ $|$y$|$ $<$ 1.2(right)
\end{itemize}
\end{textblock}
\begin{textblock}{48}(1,6)
\hspace*{10pt} \includegraphics[width=0.45\textwidth]{\FittingRltsOneS PR_2.5sigma/fractionErr_rap1.pdf}
\hspace*{10pt} \includegraphics[width=0.45\textwidth]{\FittingRltsOneS PR_2.5sigma/fractionErr_rap2.pdf}
\end{textblock}
\end{frame}

%%%%%%%%%%%%%%% 2S
\begin{frame}[t]{\small \bf fractions in P region, \structure{$\psi$(2S), nSigma = 2.0} }{}
\begin{textblock}{48}(1,5.25)
\begin{itemize}
\scriptsize \item  $|$y$|$ $<$ 0.6(left), 0.6 $<$ $|$y$|$ $<$ 1.2(right), and 1.2 $<$ $|$y$|$ $<$ 1.8(bottom)
\end{itemize}
\end{textblock}
\begin{textblock}{48}(1,6)
\hspace*{10pt} \includegraphics[width=0.45\textwidth]{\FittingRltsTwoS PR_2sigma/fraction_rap1.pdf}
\hspace*{10pt} \includegraphics[width=0.45\textwidth]{\FittingRltsTwoS PR_2sigma/fraction_rap2.pdf}
\begin{textblock}{48}(1,21)
\hspace*{10pt} \includegraphics[width=0.45\textwidth]{\FittingRltsTwoS PR_2sigma/fraction_rap3.pdf}
\end{textblock}
\end{textblock}
\end{frame}

%%%%%%%%%%% errors on fraction
\begin{frame}[t]{\small \bf fractions in P region, \structure{Errors $\psi$(2S), nSigma = 2.0} }{}
\begin{textblock}{48}(1,5.25)
\begin{itemize}
\scriptsize \item $|$y$|$ $<$ 0.6(left), 0.6 $<$ $|$y$|$ $<$ 1.2(right), and 1.2 $<$ $|$y$|$ $<$ 1.8(bottom)
\end{itemize}
\end{textblock}
\begin{textblock}{48}(1,6)
\hspace*{10pt} \includegraphics[width=0.45\textwidth]{\FittingRltsTwoS PR_2sigma/fractionErr_rap1.pdf}
\hspace*{10pt} \includegraphics[width=0.45\textwidth]{\FittingRltsTwoS PR_2sigma/fractionErr_rap2.pdf}
\begin{textblock}{48}(1,21)
\hspace*{10pt} \includegraphics[width=0.45\textwidth]{\FittingRltsTwoS PR_2sigma/fractionErr_rap3.pdf}
\end{textblock}
\end{textblock}
\end{frame}


%%%%%%%%%%% NP region
\section{Evaluate $\lpsi$ cut (NP region)} 
\begin{frame}[t]{\small \bf Evaluate $\lpsi$ cut (NP region)}{}
\begin{textblock}{48}(1,5.25)
\begin{itemize}
\scriptsize \item I) define NP region as window [nSigma*$\sigma$, +$\infty$]
\scriptsize \item II) Evaluate Prompt, Non-prompt, and Background fractions in this NP region
\end{itemize}
\end{textblock}
%\begin{textblock}{48}(1,13)
%\scriptsize sigma changes vs. $p_{T}$ for all rapidity bins: \structure{1S (left), 2S (right)}
%\hspace*{10pt} \includegraphics[width=0.45\textwidth]{\FittingRltsOneS PR_2.5sigma/rms.pdf}
%\hspace*{10pt} \includegraphics[width=0.45\textwidth]{\FittingRltsTwoS PR_2sigma/rms.pdf}
%\end{textblock}
\end{frame}

%%%%%% 1S
\begin{frame}[t]{\small \bf fractions in NP region, \structure{$\psi$(1S), nSigma = 2.5} }{}
\begin{textblock}{48}(1,5.25)
\begin{itemize}
\scriptsize \item  $|$y$|$ $<$ 0.6(left) and 0.6 $<$ $|$y$|$ $<$ 1.2(right)
\end{itemize}
\end{textblock}
\begin{textblock}{48}(1,6)
\hspace*{10pt} \includegraphics[width=0.45\textwidth]{\FittingRltsOneS NP_2.5sigma/fraction_rap1.pdf}
\hspace*{10pt} \includegraphics[width=0.45\textwidth]{\FittingRltsOneS NP_2.5sigma/fraction_rap2.pdf}
\end{textblock}
\end{frame}

%%%%%%%%%% error on fraction
\begin{frame}[t]{\small \bf fractions in NP region, \structure{Errors $\psi$(1S), nSigma = 2.5} }{}
\begin{textblock}{48}(1,5.25)
\begin{itemize}
\scriptsize \item  errors, $|$y$|$ $<$ 0.6(left) and 0.6 $<$ $|$y$|$ $<$ 1.2(right)
\end{itemize}
\end{textblock}
\begin{textblock}{48}(1,6)
\hspace*{10pt} \includegraphics[width=0.45\textwidth]{\FittingRltsOneS NP_2.5sigma/fractionErr_rap1.pdf}
\hspace*{10pt} \includegraphics[width=0.45\textwidth]{\FittingRltsOneS NP_2.5sigma/fractionErr_rap2.pdf}
\end{textblock}
\end{frame}

%%%%%%% 2S
\begin{frame}[t]{\small \bf fractions in NP region, \structure{$\psi$(2S), nSigma = 2.0} }{}
\begin{textblock}{48}(1,5.25)
\begin{itemize}
\scriptsize \item  $|$y$|$ $<$ 0.6(left), 0.6 $<$ $|$y$|$ $<$ 1.2(right), and 1.2 $<$ $|$y$|$ $<$ 1.8(bottom)
\end{itemize}
\end{textblock}
\begin{textblock}{48}(1,6)
\hspace*{10pt} \includegraphics[width=0.45\textwidth]{\FittingRltsTwoS NP_2sigma/fraction_rap1.pdf}
\hspace*{10pt} \includegraphics[width=0.45\textwidth]{\FittingRltsTwoS NP_2sigma/fraction_rap2.pdf}
\begin{textblock}{48}(1,21)
\hspace*{10pt} \includegraphics[width=0.45\textwidth]{\FittingRltsTwoS NP_2sigma/fraction_rap3.pdf}
\end{textblock}
\end{textblock}
\end{frame}

%%%%% Error
\begin{frame}[t]{\small \bf fractions in NP region, \structure{Errors $\psi$(2S), nSigma = 2.0} }{}
\begin{textblock}{48}(1,5.25)
\begin{itemize}
\scriptsize \item errors, $|$y$|$ $<$ 0.6(left), 0.6 $<$ $|$y$|$ $<$ 1.2(right), and 1.2 $<$ $|$y$|$ $<$ 1.8(bottom)
\end{itemize}
\end{textblock}
\begin{textblock}{48}(1,6)
\hspace*{10pt} \includegraphics[width=0.45\textwidth]{\FittingRltsTwoS NP_2sigma/fractionErr_rap1.pdf}
\hspace*{10pt} \includegraphics[width=0.45\textwidth]{\FittingRltsTwoS NP_2sigma/fractionErr_rap2.pdf}
\begin{textblock}{48}(1,21)
\hspace*{10pt} \includegraphics[width=0.45\textwidth]{\FittingRltsTwoS NP_2sigma/fractionErr_rap3.pdf}
\end{textblock}
\end{textblock}
\end{frame}

\end{document}
